\documentclass[a4paper,oneside]{article}
\usepackage{geometry}
\usepackage{doc}
\usepackage[latin1]{inputenc}
\usepackage[catalan]{babel}
\usepackage{amsfonts}
\usepackage{graphicx}
\usepackage{listings}
\usepackage{fancyvrb}
\usepackage{url} 
\usepackage{color}

\usepackage[pdfauthor={Ramon Xuriguera Albareda},%
		pdfsubject={BibTeX Bibliography Index Maker},%
		pdftitle={BibTeX Bibliography Index Maker: Meeting notes},%
		pdftex]{hyperref}

\lstset{%
    numbers=none,               %
    breaklines=true,            %
    fancyvrb=false,             %
    tabsize=2,                  % sets default tabsize to 2 spaces
    captionpos=b,               % sets the caption-position to bottom
    frame=single,
    xleftmargin=3em,
    xrightmargin=3em,
    backgroundcolor = \color{lightgrey}
}        


\title{BibTeX Bibliography Index Maker: Meeting Notes}
\author{Ramon Xuriguera}
\date{16-03-2010}

\setlength{\parindent}{0in}
\definecolor{lightgrey}{gray}{0.85}

\begin{document}
\maketitle

\section{Comparaci� entre motors de cerca}
Per tal d'evitar que Google ens bloquegi he estat provant altres motors de cerca. A continuaci� es mostren els resultats obtinguts:

La seg�ent taula mostra els resultats obtinguts al processar un directori amb 47 fitxers PDF. D'aquests 47, n'hi ha 5 dels quals no es pot extreure el contingut. Els percentatges estan donats en funci� d'aquests $47-5=42$ fitxers:

\begin{center}
    \begin{tabular}{|l|r|r|r|r|r|r|r|r|}
    \hline
	&	\multicolumn{2}{|c|}{Results} 	&	\multicolumn{2}{|c|}{Extracted}	&	\multicolumn{2}{|c|}{Valid}	&	\multicolumn{2}{|c|}{Invalid}	\\
    \hline
	                &	Total	&	\%	&	Total	&	\%	&	Total	&	\%	&	Total	&	\% \\
    \hline
    \multicolumn{9}{|l|}{Google}\\	
    \hline
    \hline
    \{6,10\} skip 3	&	35	&	83.33\%	&	33	&	78.57\%	&	23	&	54.76\%	&	10	&	23.81\%\\
    \{8,12\} skip 3	&	34	&	80.95\%	&	30	&	71.43\%	&	21	&	50.00\%	&	9	&	21.43\%\\
    \{10,15\} skip 0	&	35	&	83.33\%	&	34	&	80.95\%	&	23	&	54.76\%	&	11	&	26.19\%\\
    \hline
    \multicolumn{9}{|l|}{Scholar}\\	
    \hline
    \hline
    \{6,10\} skip 3	&	33	&	78.57\%	&	28	&	66.67\%	&	20	&	47.62\%	&	8	&	19.05\%\\
    \{8,12\} skip 3	&	35	&	83.33\%	&	31	&	73.81\%	&	23	&	54.76\%	&	8	&	19.05\%\\
    \{10,15\} skip 0	&	34	&	80.95\%	&	28	&	66.67\%	&	21	&	50.00\%	&	7	&	16.67\%\\
    \hline
    \multicolumn{9}{|l|}{Bing}\\	
    \hline
    \hline
    \{6,10\} skip 3	&	35	&	83.33\%	&	34	&	\textbf{80.95\%}	&	16	&	38.10\%	&	18	&	42.86\%\\
    \{8,12\} skip 3	&	36	&	\textbf{85.71\%}	&	28	&	66.67\%	&	16	&	38.10\%	&	12	&	28.57\%\\
    \{10,15\} skip 0	&	35	&	83.33\%	&	30	&	71.43\%	&	17	&	40.48\%	&	13	&	30.95\%\\
    \hline
    \multicolumn{9}{|l|}{Yahoo}\\	
    \hline
    \hline
    \{6,10\} skip 3	&	28	&	66.67\%	&	25	&	59.52\%	&	14	&	33.33\%	&	11	&	26.19\%\\
    \{8,12\} skip 3	&	24	&	57.14\%	&	23	&	54.76\%	&	10	&	23.81\%	&	13	&	30.95\%\\
    \{10,15\} skip 0	&	29	&	69.05\%	&	25	&	59.52\%	&	14	&	33.33\%	&	11	&	26.19\%\\
	    &		&		&		&		&		&		&		&	\\
    \textbf{Max}	&		&	\textbf{85.71\%}	&		&	80.95\%	&		&	54.76\%	&		&	16.67\%\\
    \hline
    \end{tabular}
    \end{center}

La resta de par�metres per fer la cerca s'ha deixat fixa: \texttt{max\_queries\_to\_try = 5} i \texttt{too\_many\_results = 25}. Modificant aquests par�metres tamb� s'obtenen variacions, per� les proves mostren que 5 i 25 permeten obtenir prou bons resultats.
\\
\\
El temps d'execuci� mig per processar aquests 47 fitxers �s d'uns 40 segons.


\subsection{Es poden millorar aquests resultats?}
S� i no. Dels 42 fitxers pels quals es pot extreure el contingut del PDF, n'hi ha 3 pels quals el contingut s'extreu, per� no t� sentit. Per exemple, tenim les seg�ents consultes:
\begin{center}
\begin{lstlisting}
Query: "rgbGphihgfBc q ed H a CFQ SV S P 6 H C8"
Query: "BA I8GF6D3 C  2 (c) A 3  (c) 5"
Query: "h c P u X hT Qy hT XH h H"
\end{lstlisting}
\end{center}
Caldr� comptar amb una reducci� del percentatge d'extraccions deguda a aquest fen�men.

\paragraph{}
El factor m�s influent per comparar els resultats dels diferents cercadors �s el n�mero de fitxers pels quals s'han obtingut resultats. El n�mero de refer�ncies (v�lides i no v�lides) tamb� �s important, per� dep�n molt dels \textit{wrappers} dels que es disposa.

Caracter�stiques de Bing:
\begin{itemize}
    \item{}
    Consultes il�limitades i els termes i condicions permeten reordenar els resultats. 
    \item{}
    Els resultats s'obtenen en XML (tot i que aix� nom�s importa a l'hora d'implementar el \textit{searcher})
    \item{}
    Pels resultats de portal.acm d�na prioritat a la p�gina de l'autor per sobre la p�gina de la refer�ncia. (Moltes vegades nom�s surt la de l'autor)
    \item{}
    No hi ha els resultats d'SpringerLink.
    
\end{itemize}

Microsoft Academic Search 

\section{Base de Dades}
Finalment s'utilitza SQLite amb SQLAlchemy, un toolkit SQL i ORM per a Python que ofereix un nivell m�s d'abstracci� per poder treballar amb bases de dades diferents sense haver de canviar el codi.
\\
\\
El m�dul corresponent a la base de dades necessita una mica de refactoring.

\section{Bloqueig ACM}
Moltes consultes provoquen que portal.acm bloqui la ip i retorni l'error 403 (Forbidden). 


\section{Informe del projecte}
\textit{No m�s tard de tres mesos (dos mesos) abans de la defensa del projecte, l'estudiant ha de presentar un informe del projecte als membres del tribunal, tal i com s'especifica a la normativa. Aquest informe ha de portar el vist-i-plau del director/ponent i �s molt convenient que l'estudiant el presenti personalment i en m� als membres del tribunal.}

\end{document}


