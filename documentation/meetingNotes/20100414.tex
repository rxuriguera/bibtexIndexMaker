\documentclass[a4paper,oneside]{article}
\usepackage{geometry}
\usepackage{doc}
\usepackage[latin1]{inputenc}
\usepackage[catalan]{babel}
\usepackage{amsfonts}
\usepackage{graphicx}
\usepackage{listings}
\usepackage{fancyvrb}
\usepackage{url} 
\usepackage{color}
\usepackage{lscape}

\usepackage[pdfauthor={Ramon Xuriguera Albareda},%
		pdfsubject={BibTeX Bibliography Index Maker},%
		pdftitle={BibTeX Bibliography Index Maker: Meeting notes},%
		pdftex]{hyperref}

\lstset{%
    numbers=none,               %
    breaklines=true,            %
    fancyvrb=false,             %
    tabsize=2,                  % sets default tabsize to 2 spaces
    captionpos=b,               % sets the caption-position to bottom
    frame=single,
    xleftmargin=3em,
    xrightmargin=3em,
    backgroundcolor = \color{lightgrey}
}        


\title{\BibTeX{} Bibliography Index Maker: Meeting Notes}
\author{Ramon Xuriguera}
\date{14-04-2010}

\setlength{\parindent}{0in}
\definecolor{lightgrey}{gray}{0.85}

\begin{document}
\maketitle

\section{Generaci� de regles}
El principal problema que he trobat un cop implementada la generaci� de regles la informaci� repetida dins de la mateixa p�gina: com m�s vegades apareix la informaci� que busquem dins de la p�gina, m�s probable �s agafar una de les etiquetes que no ens convenen. Per exemple, en el cas de la p�gina \textit{Science Direct}, el t�tol de l'article apareix en el t�tol de la p�gina, juntament amb tro�os d'informaci� que van canviant. Aix� fa que la regla resultant no sigui tant bona com la que es podria haver obtingut fent servir una de les altres aparicions del t�tol dins la p�gina:
\begin{center}
\begin{lstlisting}
Path: [[[u'title', {}, 5]]]
Regex: u'ScienceDirect\ \-\ (?:.*)e\ (?:.*)\ \:\ (.*)(?:.*)\ '
\end{lstlisting}
\end{center}

Enlloc de:
\begin{center}
\begin{lstlisting}
Path: [[[u'div', {u'id':'articleTitle'}, 0]]]
Regex: u'(.*)'
\end{lstlisting}
\end{center}

Aquest problema afecta molt en el cas dels anys, sobretot en articles nous on l'any tamb� pot apar�ixer al copyright de la p�gina, etc.

\section{Emmagatzemar wrappers a la base de dades}
Fet

\section{Tasques pendents}
Llistat de tasques pendents a realitzar:
\begin{itemize}
\item{}
Netejar, encara m�s, l'HTML abans de fer l'extracci�: treure comentaris i etiquetes \texttt{script}, \texttt{style}, etc.
\item{}
Afegir els camps especials com ara autors a la generaci� de wrappers.
\item{}
Comprovaci� de l'estat dels wrappers actuals. (Comparar timestamps)
\item{}
Interf�cie
\end{itemize}
\end{document}


