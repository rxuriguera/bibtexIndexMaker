\documentclass[a4paper,oneside]{article}
\usepackage{geometry}
\usepackage[latin1]{inputenc}
\usepackage[catalan]{babel}
\usepackage{amsfonts}
\usepackage{url} 

\usepackage[pdfauthor={Ramon Xuriguera Albareda},%
		pdfsubject={BibTeX Bibliography Index Maker},%
		pdftitle={BibTeX Bibliography Index Maker: Notes},%
		pdftex]{hyperref}
		

\title{BibTeX Bibliography Index Maker: Notes}
\author{Ramon Xuriguera}
\date{}

\begin{document}
\maketitle

\section{BibTeX}
Aspectes del format BibTeX a tenir en compte:
\begin{itemize}
\item{Com podem distingir entre diferents tipus d'entrada (article, book, inproceedings, etc.) a partir del fitxer?}
\item{Format dels noms. Un nom consisteix de diferents parts: First, von, Last, Jr. El token \textit{von} o \textit{de la} cal posar-los
en min�scules. Per tal que el nom es reconegui, cal que tingui el format: von Last, Jr, First. D'aquesta manera, si hi ha m�s d'un
cognom no passa res.}
\item{Car�cters Unicode entre claus per poder ser utilitzats correctament per l'estil \textit{alpha}. Per exemple \verb=Jos{\'{e}=}
\item{Per prevenir que BibTeX canvi� un text a min�scules, cal posar el text entre claus.}
\item{Si hi ha massa autors, truncar la llista amb \textit{et al.}}
\item{Utilitzar abreviatures de tres lletres per als mesos}
\item{Utilitzar el camp \texttt{key} per a organitzacions amb un nom llarg, de manera que s'utilitzin les inicials de l'organitzaci� 
al fer una cita}

\end{itemize}

\section{PDF}
Si un fitxer PDF ha estat generat utilitzant el paquet \texttt{hyperref} o similar, podem obtenir les seg�ents dades del fitxer sense
haver-ne d'extreure el text: n�mero de p�gines, t�tol, autor, assumpte i paraules clau. Si aquests camps no s'han omplert al 
generar el fitxer, estaran en blanc.
\\
\\
Aspectes a considerar:
\begin{itemize}
\item{Car�cters Unicode}
\item{Glyphs com ara \textit{fi} corresponen a m�s d'una lletra}
\end{itemize}

\subsection{Text extraction}
\subsubsection{PDFBox}
Aquest projecte es troba a la incubadora d'Apache.
\subsubsection{pyPDF}
\subsubsection{PDFMiner}
Programes desenvolupats en Python per etreure i analitzar dades de documents PDF.
\subsubsection{jPod}

\section{Articles Db}
\subsection{DBLP}
DBLP proporciona un 
\subsection{Portal ACM}
\subsection{Altres}
\begin{itemize}
\item{CiteSeerX}

\item{Arxiv.org}
	\url{http://arxiv.org/help/api/index}
\end{itemize}

\section{Software existent}
\url{http://en.wikipedia.org/wiki/Comparison_of_reference_management_software}
\subsection{JabRef}
\begin{itemize}
\item{Java}
\item{Llic�ncia: GPL}
\item{�s possible crear plug-ins amb \textit{Java Plug-in Framework}}
\item{\url{http://jabref.sourceforge.net/}}
\end{itemize}

\subsection{Mendeley}
\begin{itemize}
\item{Propietari}
\item{No accepta plug-ins}
\item{\url{http://www.mendeley.com/}}
\end{itemize}

\subsection{BibDesk}
\begin{itemize}
\item{C, Objective C}
\item{Llic�ncia: BSD}
\item{\url{http://bibdesk.sourceforge.net/}}
\end{itemize}

\end{document}