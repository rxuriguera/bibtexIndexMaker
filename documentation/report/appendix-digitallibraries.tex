\chapter{Biblioteques digitals}
Al llarg d'aquest document hem definit qu� s�n les biblioteques digitals i tamb� n'hem anat esmentant algunes de les que hi ha disponibles a Internet. En aquest ap�ndix nom�s intentem llistar i facilitar enlla�os a totes les que s'han utilitzat a l'hora de realitzar les proves. Com a curiositat, tamb� hem incl�s petites introduccions extretes de les descripcions escrites per ells mateixos:

\begin{itemize}
\item{ACM Portal:}\textit{
Full text collection of every article published by ACM, including over 50 years of archives.
}
\\
\url{http://portal.acm.org}



\item{CiteSeerX:}\textit{
is a scientific literature digital library and search engine that focuses primarily on the literature in computer and information science. 
}
\\
\url{http://citeseerx.ist.psu.edu}



\item{CiteULike:}\textit{
is a free service for managing and discovering scholarly references.
}
\\
\url{http://www.citeulike.org}



\item{EconPapers:}\textit{
EconPapers use the RePEc bibliographic and author data, providing access to the largest collection of online Economics working papers and journal articles.
}
\\
\url{http://econpapers.repec.org}



\item{IDEAS:}\textit{
is a service providing information about working papers and published research to the economics profession. IDEAS stands for ``Internet Documents in Economics Access Service'', which is not very good English, but you get the idea... 
}
\\
\url{http://ideas.repec.org}



\item{IEEE Computer Society:}\textit{
With nearly 85,000 members, the IEEE Computer Society is the world's leading organization of computing professionals. [...] the Computer Society is dedicated to advancing the theory and application of computing and information technology.
}
\\
\url{http://www.computer.org}



\item{Informa World:}\textit{
is the leading provider of specialist information to the global academic \& scientific, professional and commercial communities via publishing, events and performance improvement.
}
\\
\url{http://www.informaworld.com}



\item{ScientificCommons:}\textit{
aims to provide the most comprehensive and freely available access to scientific knowledge on the internet. The major aim of the project is to develop the world's largest communication medium for scientific knowledge products which is freely accessible to the public.
}
\\
\url{http://en.scientificcommons.org}



\item{ScienceDirect:}\textit{
is a leading full-text scientific database offering journal articles and book chapters from more than 2,500 peer-reviewed journals and more than 11,000 books.
}
\\
\url{http://www.sciencedirect.com}



\item{SpringerLink:}\textit{
One of the world's leading interactive databases for high-quality STM journals, book series, books, reference works and the Online Archives Collection. SpringerLink is a powerful central access point for researchers and scientists.
}
\\
\url{http://springerlink.com}
\end{itemize}

