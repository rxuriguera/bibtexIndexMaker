\chapter{Introducci�}
\label{introduction}

\section{Descripci�}
\textit{\BibTeX{} Bibliography Index Maker} �s una eina d'ajuda a la creaci� d'�ndexs bibliogr�fics pensada com un complement a aplicacions de maneig de refer�ncies ja existents com poden ser \textit{\href{http://jabref.sourceforge.net/}{JabRef}}\footnote{\href{http://jabref.sourceforge.net/}{http://jabref.sourceforge.net}} o \textit{\href{http://www.mendeley.com/}{Mendeley}}\footnote{\href{http://www.mendeley.com/}{http://www.mendeley.com}}.

\paragraph{}
La principal funcionalitat consisteix en escanejar un directori que cont� articles cient�fics en PDF i generar un �ndex bibliogr�fic en \BibTeX{} amb les refer�ncies d'aquests fitxers. Aquest �ndex es pot importar des de les aplicacions esmentades o b� pot ser referenciat directament des d'un nou document \TeX.

\section{Treball Existent}
Actualment existeixen nombroses aplicacions dedicades al maneig de refer�ncies. Algunes d'elles utilitzen les meta-dades dels fitxers per tal de trobar informaci� com ara el t�tol o l'autor, per� cap de les eines que hem trobat llegeix el contingut dels documents.

\paragraph{}
Empreses com ara \textit{Google} o \textit{Microsoft} agafen la informaci� de documents PDF per oferir serveis com ara \textit{Scholar} o \textit{Academic Search}, per� no ofereixen el codi font i per tant, no sabem com funcionen. \textit{CiteSeer} �s un projecte \textit{open source} de caracter�stiques similars, per� que tamb� t� limitacions. El sistema funciona analitzant la bibliografia dels articles, per� t� problemes per obtenir els camps de la cap�alera del propi fitxer, que �s el que ens interessa.

