\chapter{Extracci� dels continguts d'un PDF}
\label{pdftotext}
La primera idea a l'hora d'abordar el nostre projecte va ser intentar extreure informaci� directament dels fitxers PDF dels quals es disposa.

\section{Dificultats}
Les principals dificultats s�n:
\begin{itemize}
\item{Car�cters especials: } com Unicode o lligadures
\item{Flux del text dins del fitxer}
\end{itemize}

\section{Programari existent}
Tot hi haver-hi diverses utilitats que permeten l'extracci� del contingut d'un fitxer PDF en forma de text pla o HTML, totes presenten problemes similars en els punts comentats a la secci� anterior.

A l'ap�ndix \ref{appendix-pdf2text} hi ha exemples de com queden els texts extrets de diferents documents PDF.


\textbf{xPDF} proporciona eines executables des de la l�nia de comandes per extreure text i altres elements dels fitxers PDF. Es distribueixen binaris de la utilitat tant per Windows com per Linux (que tamb� funcionen per MAC OS).
El principal motiu pel qual hem escollit xPDF �s la qualitat dels resultats, en especial, el fet que no separa els par�grams en diferents l�nies i que obt� el text segons l'ordre de lectura i no l'ordre en que es troben en el document (e.g. dues columnes). Tamb� ser� �til la possibilitat d'extreure les metadades del fitxer de forma f�cil.

Altres opcions que s'han tingut en compte:
\begin{itemize}
\item{PyPDF}
\item{PDFMiner}
\item{PDFBox}
\end{itemize}

 
