\chapter{Cerca de refer�ncies a Internet}
\label{ir}


\section{Cercadors}
La primera idea per cercar p�gines que contenen refer�ncies d'art�cles va ser utilitzar \textit{Google Scholar}. La falta d'APIs i el bloqueig peri�dic de les cerques autom�tiques van fer 
Hem preparat el nostre cercador per tal d'utilitzar les APIs dels cercadors \textit{Google}, \textit{Yahoo} i \textit{Bing} i hem 

El principal avantatge �s la 

Un inconvenient, hi ha biblioteques virtuals que no estan indexades en aquests serveis.


\section{Ajustaments}
Podem ajustar la manera com es fan les cerques a partir de certs par�metres que es detallen a continuaci�.

En moltes ocasions, el cercador \textit{Bing} mostra resultats corresponents a \textit{Microsoft Academic Search} (un projecte molt similar a  \textit{Google Scholar}). Aquestes p�gines, per�, no mostren prou informaci� com per generar refer�ncies. Per tant, les hem d'ometre.

\section{\textit{Multithreading}}
Un dels inconvenients m�s grans que implica el fet d'haver d'accedir a Internet, �s que el temps perdut esperant dades �s molt alt. Per reduir-lo, s'ha estudiat la possibilitat d'utilitzar diferents fils d'execuci� per fer m�s d'una consulta de forma m�s o menys simult�nia. La taula seg�ent mostra una comparativa del temps necessari per obtenir m�tliples p�gines web de forma seq�encial o b� utilitzant fins a cinc fils d'execuci� diferents. 

    \begin{center}
    \begin{tabular}{|r|r|r|r|r|r|r|r|}
        \hline
        \multicolumn{2}{|c|}{2 p�gines} & \multicolumn{2}{|c|}{5 p�gines} & \multicolumn{2}{|c|}{10 p�gines} & \multicolumn{2}{|c|}{20 p�gines} \\
        \hline
        Seq. & 5 Threads          & Seq. & 5 Threads          & Seq. & 5 Threads           & Seq. & 5 Threads \\
        \hline
        \hline
        0.9010 & 0.5481 & 2.1830 & 0.6612 & 4.3153 & 1.5914 & 7.9295 & 2.5949 \\
        0.7467 & 0.3795 & 2.1558 & 0.7441 & 4.3186 & 1.2311 & 8.5483 & 2.1958 \\ 
        0.7678 & 0.5641 & 2.0645 & 0.5383 & 9.2930 & 1.4415 & 8.7202 & 2.5749 \\
        0.7421 & 0.3876 & 2.0684 & 0.8551 & 4.9859 & 1.5294 & 8.4732 & 2.2841 \\
        0.9674 & 0.5477 & 2.1510 & 0.8550 & 5.3600 & 1.3116 & 9.2901 & 2.2257 \\
        \hline
        \multicolumn{8}{|l|}{Mitjana:} \\
        \hline
        0.8250 & 0.4854 & 2.1246 & 0.7307 & 5.6546 & 1.4210 & 8.5923 & 2.3751 \\
        \hline
        \multicolumn{8}{|l|}{Guany:} \\
        \hline
        \multicolumn{2}{|c|}{\textbf{-44.96\%}} &  \multicolumn{2}{|c|}{\textbf{-65.6\%}} &  \multicolumn{2}{|c|}{\textbf{-74.87\%}} &  \multicolumn{2}{|c|}{\textbf{-72.35\%}} \\
        \hline
    \end{tabular}
    \end{center}

Les p�gines p�gines corresponen a consultes aleat�ries a Google per evitar l'efecte dels \textit{proxies} i la mem�ria \textit{cache}. Com a conclusi�, tot i que es dades obtingudes no s�n riguroses, ens donen una idea for�a clara de la millora que s'obt� utilitzant aquesta t�cnica.
\paragraph{}
Al nostre sistema hem implementat un \textit{pool} amb un n�mero variable de fils d'execuci� que es van reutilitzant mentre queden refer�ncies per extreure.

