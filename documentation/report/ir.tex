\chapter{Cerca de refer�ncies a Internet}
\label{ir}

\section{Primera idea: \textit{Google Scholar}}

\section{Resta de cercadors}
Hem preparat el nostre cercador per tal d'utilitzar les APIs dels cercadors \textit{Google}, \textit{Yahoo} i \textit{Bing} i hem 

El principal avantatge �s la 

Un inconvenient, hi ha biblioteques virtuals que no estan indexades en aquests serveis.


\section{Ajustaments}
Podem ajustar la manera com es fan les cerques a partir de certs par�metres que es detallen a continuaci�.

En moltes ocasions, el cercador \textit{Bing} mostra resultats corresponents a \textit{Microsoft Academic Search} (un projecte molt similar a  \textit{Google Scholar}). Aquestes p�gines, per�, no mostren prou informaci� com per generar refer�ncies. Per tant, les hem d'ometre.

\section{\textit{Multithreading}}
Un dels inconvenients que suposa el fet d'haver d'accedir a Internet, �s la lat�ncia. Per reduir el temps que es perd esperant les dades, hem fet que l'aplicaci� cre� diferents fils d'execuci�.

