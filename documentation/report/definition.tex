\chapter{Definici� del Projecte}
\label{definition}
\section{Context}
Phasellus eu ante diam, eu euismod nunc. Vivamus non dolor sem. Sed id metus enim. Curabitur consectetur eleifend quam porta sagittis. Mauris sed augue fermentum leo pharetra posuere nec euismod risus. In dui elit, iaculis eget vestibulum eu, suscipit at purus. Mauris hendrerit condimentum velit, in facilisis dui consectetur non. Quisque tristique velit vitae enim posuere suscipit. Integer condimentum rutrum accumsan. Suspendisse bibendum urna eget orci aliquam faucibus congue urna consequat. Nam elementum, lectus a volutpat gravida, felis nibh faucibus nibh, id fringilla arcu purus sed orci.

\section{El format \BibTeX}
\begin{itemize}
\item{Com podem distingir entre diferents tipus d'entrada (article, book, inproceedings, etc.) a partir del fitxer?}
\item{Format dels noms. Un nom consisteix de diferents parts: First, von, Last, Jr. El token \textit{von} o \textit{de la} cal posar-los
en min�scules. Per tal que el nom es reconegui, cal que tingui el format: von Last, Jr, First. D'aquesta manera, si hi ha m�s d'un
cognom no passa res.}
\item{Car�cters Unicode entre claus per poder ser utilitzats correctament amb l'estil \textit{alpha}. Per exemple: \verb=Jos{\'{e}}=}
\item{Per prevenir que BibTeX canvi� un text a min�scules, cal posar el text entre claus.}
\item{Si hi ha massa autors, truncar la llista amb \textit{et al.}}
\item{Utilitzar abreviatures de tres lletres per als mesos}
\item{Utilitzar el camp \texttt{key} per a organitzacions amb un nom llarg, de manera que s'utilitzin les inicials de l'organitzaci� 
al fer una cita.}
\end{itemize}

\section{Caracter�stiques}
At vero eos et accusamus et iusto odio dignissimos ducimus qui blanditiis praesentium voluptatum deleniti atque corrupti quos dolores et quas molestias excepturi sint occaecati cupiditate non provident, similique sunt in culpa qui officia deserunt mollitia animi, id est laborum et dolorum fuga. Et harum quidem rerum facilis est et expedita distinctio.

\section{Planificaci� Temporal}

