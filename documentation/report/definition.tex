\chapter{Definici� del Projecte}
\label{definition}
\section{Context}


\section{\BibTeX}
Per poder entendre el context del projecte cal que descrivim l'eina de maneig de refer�ncies \BibTeX i la sintaxi del llenguatge que utilitza. 
En el nostre cas farem servir aquest llenguatge com a format de sortida al generar els �ndexos bibliogr�fics. Al llistat \ref{listing:exampleBibTeX} es mostra un exemple d'una refer�ncia d'un article cient�fic expressat en el format \BibTeX:
\begin{center}
\begin{lstlisting}[caption={Refer�ncia expressada en \BibTeX}, label=listing:exampleBibTeX]
@article{MoSh:27,
  title = {Size direction games over the real line},
  author = {Moran, Gadi and Shelah, M., Saharon},
  journal = {Israel Journal of Mathematics},
  pages = {442--449},
  volume = {14},
  year = {1973},
}
\end{lstlisting}
\end{center}

Alguns aspectes a comentar sobre l'exemple anterior:
\begin{itemize}
\item{}
La primera l�nia cont� el tipus de document i un identificador. El primer defineix els camps obligatoris que s'han d'especificar, i el segon ens permetr� citar a la refer�ncia des d'un document. En el nostre cas nom�s ens interessen les refer�ncies de tipus \textit{article} i haurem de definir, com a m�nim, els camps: \textit{author}, \textit{title}, \textit{journal} i \textit{year}.

\item{}
Es considera que el nom d'un autor o editor pot constar de quatre parts diferents: \textit{First}, \textit{von}, \textit{Last}, \textit{Jr.}. Es poden ordenar de diverses maneres, per� nosaltres ho farem amb \texttt{<von> <last>, <middle>, <first>}. Cal separar m�ltiples noms amb la paraula \texttt{and}.

\item{}
L'�ltim camp d'una refer�ncia pot acabar o no amb una coma.
\end{itemize}

\section{Caracter�stiques}


\section{Planificaci� Temporal}

