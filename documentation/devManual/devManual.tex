\documentclass[a4paper,oneside]{article}
\usepackage{geometry}
\usepackage[latin1]{inputenc}
\usepackage[catalan]{babel}
\usepackage{amsfonts}
\usepackage{graphicx}
\usepackage{fancyvrb}
\usepackage{listings}
\usepackage{hyphenat}
\usepackage{url} 

\usepackage[pdfauthor={Ramon Xuriguera Albareda},%
		pdfsubject={BibTeX Bibliography Index Maker},%
		pdftitle={BibTeX Bibliography Index Maker: Developer Manual},%
		pdftex]{hyperref}

\lstset{%
    numbers=none,               %
    breaklines=true,            %
    fancyvrb=false,             %
    tabsize=2,	                % sets default tabsize to 2 spaces
    language=sh,                %
    captionpos=b                % sets the caption-position to bottom
}		

\newcommand{\bimhome}{\$BIBIM\_HOME }


\title{BibTeX Bibliography Index Maker: Developer Manual}
\author{Ramon Xuriguera}
\date{}

\begin{document}
\maketitle

\section{Setting up the environment}
\begin{itemize}
    \item{}
    Make sure that your box fulfils the following software requirements:
    \begin{itemize}
        \item{Python 2.6: }
            \url{http://www.python.org/download/releases/}
        \item{Git: }
            \url{http://git-scm.com/}
    \end{itemize}
    Any other tool should be automatically downloaded and installed using the provided bootstrap scripts.

    \item{Checkout the project from \textit{github}:}
    Open a terminal, go to your enlistments folder (e.g. \texttt{/home/rxuriguera/enlistments}) and run
    \begin{lstlisting}
cd <path-to-enlistments>
git clone git://github.com/rxuriguera/bibtexIndexMaker.git <enlistment-dir>
    \end{lstlisting}
    From now on, we will refer to \texttt{<path-to-enlistments>/<enlistment-dir>} as \bimhome

    \item{}
    Run the environment set-up script:
    \begin{lstlisting}
cd bibim/infrastructure/environment
./setup-devenv
    \end{lstlisting}

    \item{}
    Create a desktop icon or a launcher that invokes \bimhome/infrastructure/enlistments/devenvlauncher. Make sure to pass \bimhome as a parameter to the script.
\end{itemize}

\section{Project Structure}
The project's source has been divided into different packages each of which uses \texttt{buildout}. They can be found at \bimhome/packages. Build scripts and the like are placed under \bimhome/infrastructure.

\section{Build Infrastructure}
The builds should be scheduled and tracked using \texttt{Buildbot}, which can be easily installed with the following code:
    \begin{lstlisting}
cd bibim/infrastructure/buildbot
./setup
./start
    \end{lstlisting}
In order to deploy \texttt{Buildbot}, \texttt{Twisted} has to be installed on your system. If you allow \texttt{buildmaster} to be accessed from the Internet, you can hook up a script to GitHub so every check-in triggers a new build.

\end{document}
